% \section{Background and Informal Presentation} \label{sec:bg}

\emph{Combinatory Reduction Systems} (CRSs), introduced by
Aczel~\cite{Aczel:78} and later refined by Klop~\cite{1980_klop}, form
a generalization of first-order term rewriting to higher-order
rewriting.  CRSs provide the theoretical basis of our compiler
generator \Tosca. It has turned out that higher-order rewriting
provides a good match for the various tasks (parsing, normalization,
specificity, and code generation) required in a compiler, and thus in
its construction.

On one hand \Tosca is a compiler generator, that is, \emph{a tool} for
constructing compilers, whose setup is depicted in
Figure~\ref{fig:use} below and detailed in Section~\ref{sec:use}. On
the other hand \Tosca is a \emph{programming language}, namely a
self-hosted, lazy, higher-order functional language, with sometimes
unique features, specific to compilers, like specificity-ordered
pattern matching, and embedded programs,
cf.~Section~\ref{sec:language}.

CRSs form the essential core of \Tosca. While practical
necessities for compiler construction naturally push certain demands
on the design (cf.~Section~\ref{sec:surface}), we have been careful to
protect this theoretical underpinning as a separate part (\Tosca
Core) of our language.

The overall setup of \Tosca strives for a light-weight
architecture. For example a typical task of a compiler is to optimize
the previously generated intermediate representation of the source
language. Hence, it may be to be expected that a similar optimization
stage is required in a compiler generator to trigger this
behavior. However, an elegant possibility is to learn from
transpilers. \Tosca merely \emph{transforms} the provided source
language \PL into a dedicated target (intermediate) source
representation.  Then an off-the-shelf compiler for this target
language (for the moment Java) handles the optimization.

The development of \Tosca has been greatly motivated by a preceding
compiler generator project based on higher-order rewriting, namely
\crsx, mainly developed by Rose~\cite{2010_rose,2011_rose}. \crsx has
been successfully applied in practice by IBM DataPower Gateway%
\footnote{cf. \href{http://www.ibm.com/datapower}{www.ibm.com/datapower}.}
to implement an XQuery~\cite{Walmsley:07} and a JSONiq~\cite{FF:13}
compiler and is also used as a back-end for the teaching interface
HACS\footnote{cf.~\href{https://github.com/crsx/hacs}{github.com/crsx/hacs}.}.\cite{gentle_hacs}

\subsection{Generating  a Compiler With \Tosca} \label{sec:use}

\Tosca's main purpose is to ease the construction of
compilers. Thus, we start by describing how to construct a compiler
with \Tosca. Given some programming language~\PL, and some
program written in \PL which we want to compile: How can we use
\Tosca to create such a compiler? We depict the overall setup in
Figure~\ref{fig:use}.
%
\begin{figure}[h]
\centering
\begin{tikzpicture}
    \node at (0,0) [box, minimum width = 2.5cm
                       , minimum height = 1cm]
                   (Syntax) {syntax of \PL \\[-0.2cm] %
                            \rule{2.3cm}{0.4pt} \\
                            \antlr};
    \node at (2.85,0) [box, minimum width = 2.65cm
                          , minimum height = 1cm]
                      (Semantic) {semantic of \PL \\[-0.2cm] %
                                 \rule{2.3cm}{0.4pt} \\
                                 \Tosca language};
    \node at (0,-1.6) [box, minimum width=5.75cm
                           , fill=gray!20]
                       (TransScript) {\Tosca compiler generator};
   \node at (5.7,-1.95) (Cogs1) {\cogs{0.15}};
    %
    \path[->, thick] (Syntax)   edge (TransScript.171)
                     (Semantic) edge (TransScript.10);
    % 
    \node at (0,-2.5) [box, minimum width=5.75cm] %
                    (Java) { \java source code for \PL compiler \\[-0.2cm] 
                             \rule{4.6cm}{0.4pt} \\
                             \java};
    \path[->, thick] (TransScript) edge (Java);
    %
    \node (Comp) at (0,-4.25) [box, minimum width=5.75cm, fill=gray!20]
                   { \PL compiler};
    \path[->, dotted, thick] (Java) edge (Comp);
    %
    \node (pl) at (-2.5,-4.25) [box,draw] { \PL program };
    \path[->] (pl) edge (Comp) 
              ;
    \node at (5.7,-4.6) (Cogs2) {\cogs{0.15}};
  \end{tikzpicture}
  \caption{Compiler Generation for Programming 
    Language~\PL.}
  \label{fig:use}
\end{figure}

%%% Local Variables:
%%% mode: latex
%%% TeX-master: "../techreport"
%%% End:

%
As can be seen on the top, to create a compiler for \PL, \Tosca
needs as input:
\begin{enumerate}
\item the \emph{syntax} of \PL in the
  \antlr~v4 format \cite{Parr:2013:DAR:2501720}, and
\item the \emph{semantics} of \PL as \Tosca rules.
\end{enumerate}
%
Based on this input \Tosca creates the source code of a compiler
for \PL. In Figure~\ref{fig:use} this execution of \Tosca is
indicated by cogwheels.
%
Currently, the generated
source code is \java~8 source code. However, the architecture of
\Tosca allows a simple integration of any suitable target
language, cf.~Section~\ref{sec:architecture}.
%
Finally, the generated \java source code has to be compiled---%
using an off-the-shelf \java compiler---to obtain
an executable \PL compiler. Observe that the \PL compiler 
itself targets a programming language and it is up to the compiler writers to 
decide which language to use (in the semantics of \PL).

We exemplify this setup with respect to a simple, strict, and
pure functional language, namely \MiniML
%
\footnote{The syntax is adapted from~\href{http://andrej.com/plzoo/html/miniml.html}{\url{andrej.com/plzoo/html/miniml.html}}. 
The full \antlr grammar is available at 
\anolink{\ToscaPath/samples/miniml/MiniML.g4}{\cd{MiniML.g4}}{\cd{samples/miniml/MiniML.g4}}.}.
%
We describe the creation of an interpreter for \MiniML, 
that is, \PL == \MiniML. First we need an \antlr grammar
describing the syntax, which we will present next. For presenting
\antlr grammars throughout this paper, we will employ the following
conventions: Non-terminals, i.e., \antlr grammar rules, are written in
lower case letters, e.g., \antlrIn{expr}. For terminal symbols it
depends. If they have a verbatim match, such as keywords and
delimiters, we inline and enclose them in \antlrIn{''}, e.g.,
\antlrIn{'let'} or \antlrIn{'='}. If they cannot be described
verbatim, as for example numbers or variables, then we write them in
capital letters, e.g., \antlrIn{VAR}. 

\begin{example} \label{ex:MiniML} %
  %
  An \antlr description of the \MiniML language:
  %
\begin{lstANTLR}
toplevel : 'let' VAR '=' expr ';;' toplevel ';;'
            | expr ';;'

expr : timesExpr '+' expr | timesExpr '-' expr
       | timesExpr

timesExpr : comprExpr '*' timesExpr | comprExpr

comprExpr : primaryExpr '<' comprExpr
           | primaryExpr '=' comprExpr
           | primaryExpr

primaryExpr : app_expr | '-' INT |
             | 'if' expr 'then' expr 'else' expr
             | 'fun' VAR '(:' ty '):' ty 'is' expr

app_expr : simple_expr | simple_expr app_expr      

simple_expr : 'true' | 'false' | '(' expr ')' 
             | INT |  VAR 

tyPrimary : 'bool' | 'int' | '(' ty ')'

ty : tyPrimary | tyPrimary '->' ty

VAR : ['a'-'z' 'A'-'Z']+
INT : ['0'-'9']+
\end{lstANTLR}
\end{example}
%
Given the syntax of \MiniML, we now need to describe the
\emph{semantics} in the \Tosca language to generate a \MiniML
interpreter. To give an idea and a glimpse of the \Tosca
language, we look at how we can describe the \antlrIn{'+'}.
%
\begin{lstTosca}
rule EvalExpr(expr⟦ #timesExpr + #expr ⟧) 
  → Plus(EvalTimesExpr(#timesExpr), EvalExpr(#expr))
\end{lstTosca}
%
To evaluate the expression \ToscaIn{#timesExpr + #expr}, we rely on
\ToscaIn{Plus} provided by a standard library in \Tosca. Here also we
already present the feature of \emph{embedded program} snippets:
within \ToscaIn{expr⟦ ⟧} we can embed \MiniML syntax---a feature
explained in detail in the next section.  Of course, we could as well
define the functionality easily ourselves. Next we look at a different
snippet: the \antlrIn{'if then else'}.
%
\begin{example} \label{ex:evalminiml} 
The conditional \antlrIn{'if then else'} is implemented as follows:

\begin{lstTosca}
rule EvalExpr(expr⟦if #expr1 then #expr2 else #expr3⟧) 
  → EvalIf(EvalBoolExpr(#expr1), #expr2, #expr3)

rule EvalIf(TRUE, #expr2, #expr3) → EvalExpr(#expr2)
rule EvalIf(FALSE, #expr2, #expr3) → EvalExpr(#expr3)

rule EvalBoolExpr(simple_expr⟦ true ⟧) → TRUE
rule EvalBoolExpr(simple_expr⟦ false ⟧) → FALSE
...
\end{lstTosca}
\end{example}

The above rules are, of course, incomplete, as indicated by the ellipses. More
of \MiniML code snippets will return throughout the next section,
where we introduce the \Tosca language. Also notice that the choice
of showing the \MiniML interpreter as opposed as the compiler is purely
driven by presentation considerations, and not technical ones.  

Naturally, \MiniML is a toy language.  But we have used \Tosca already
in a real-world application: In Section~\ref{sec:self-hosting} we
describe how we successfully used \Tosca to create a compiler for the
\Tosca language itself, that is, we can bootstrap (aka self-host)
\Tosca. I.e., the setup depicted in Figure~\ref{fig:use} is
applicable in its fullest generality, namely we have:
%
\begin{center}
  \PL == \Tosca language.
\end{center}
%

\subsection{\Tosca as a Functional Language} \label{sec:language}

This section gives an overview of the \Tosca language through
code snippets. A subset of the formal syntax is given in 
Section~\ref{sec:coretotransscript} and the complete \Tosca syntax
written as an \antlr v4 grammar can be found in 
%
\anolink{\ToscaPath/src/parser/TransScript.g4}
        {\cd{Tosca.g4}}{\cd{src/parser/TransScript.g4}}.
%
\Tosca is a higher-order, lazy functional language with good support for efficient
recursion which makes it easy to work with trees.
In particular it is very suited to work on 
one of the main data structures in compiler construction: 
\emph{Abstract Syntax Trees} (\emph{ASTs} for short). 
It naturally borrows many characteristics and features found in
existing functional programming languages, such as referential
transparency, first-class functions, and pattern matching.
%
Describing them in detail is out of scope.
Instead the syntax of the  most commonly used features is illustrated
in the example below.
%
\begin{example} \label{ex:map} %
We give the \Tosca syntax for the implementation of
\ToscaIn{Map} over a list of elements of type \ToscaIn{a}.
%
\begin{lstTosca}
func Map<a b>([a] -> b, List<a>) -> List<b>
 
rule Map([x] -> #F(x), ())          
  → ()

rule Map([x] -> #F(x), (#Yhd, #Ytl...)) 
  → (#F(#Yhd), Map([x] -> #F(x), #Ytl))
\end{lstTosca}
\end{example}
%
A \emph{function} is declared using the keyword \ToscaIn{func}
followed by its name, parameter types, and return type. In this
example, \ToscaIn{Map} is a parameterized function with two type
parameters \ToscaIn{a} and \ToscaIn{b},  
and it takes two parameters, the first
one is a function taking one argument of type \ToscaIn{a} and
returning a value of type \ToscaIn{b}, the second one is a list
of values of type \ToscaIn{b}. 
In \Tosca, function names and types start with an upper case character. 
Type parameters and variables start with a lower case character.

A \emph{rule} starts with the keyword \ToscaIn{rule}.  As usual a rule
provides a function definition for the function on the left hand by
providing a directed defining equation.
%
Two rules are associated to the \ToscaIn{Map} function
declaration.  For both rules, the first argument \ToscaIn{[x] -> #F(x)}
matches any function taking one parameter.

We emphasize that the expression \ToscaIn{[x] -> #F(x)} is used
in \Tosca code differently based on context. 
In the function declaration it plays its standard role as an arrow type. 
However, perhaps unexpectedly in the first rule it reappears. In this
latter context \ToscaIn{[x] -> #F(x)} denotes the lambda abstraction
\ToscaIn{[x]#F(x)} presented in Section~\ref{sec:CRS}. 
%
We allow for this overloading, as the arrow notation is a common
notation for closures in C-like languages. However, note that it is
merely syntactic sugar which is represented as expected in
\Tosca Core (cf.~Section~\ref{sec:coretotransscript}).

Continuing with Example~\ref{ex:map}, matching values are bound to
\emph{meta variables}.  Meta variables start in \Tosca with the
character \ToscaIn{#}, as \ToscaIn{#F, #Yhd, #Ytl} in
Example~\ref{ex:map}.  Meta variables which are bound in a pattern can
either be discarded, like in the first rule, or used in the rule
expression. The matching function in both rules is bound to the meta
variable named \ToscaIn{#F} (cf.~Section~\ref{sec:CRS}).
%
The second argument of the first rule matches the empty list, i.e.,
\ToscaIn{()}.  The second argument of the second rule matches a
list with at least one element, bound to \ToscaIn{#Yhd} and the
rest of the list, using the \ToscaIn{...} notation, is bound to
\ToscaIn{#Ytl}.

By default, rules are \emph{applied} in lexical order, from top to
bottom, unless another evaluation strategy is specified.  In the
example, the first rule matches when the second argument is the empty
list, and produces the empty list as the result.  When the second rule
matches, for non-empty list, it produces a list where the first item
is the result of the function \ToscaIn{#F} applied to the
argument \ToscaIn{#Yhd}.  The list tail is the result of
\ToscaIn{Map} applied to the rest of the list
\ToscaIn{#Ytl}. Section~\ref{sec:codegen} explains how the
default evaluation order is implemented. Sections~\ref{sec:surface}
and~\ref{sec:specification} describe howto enhance this functionality
to make it more practical.

Finally, we apply \ToscaIn{Map} on a real input. Therefore we  provide a third
  rule, a \ToscaIn{Main} function.
%
\begin{example}[continued from Example~\ref{ex:map}]
\label{ex:map:2}
  We get:
  \begin{lstTosca}
func Main -> List<Int>
rule Main 
  → Map( [x] -> Plus(1, x), (1, 2, 3) )
  \end{lstTosca}
\end{example}
%
The function \ToscaIn{Main} shows how \ToscaIn{Map} is
called with a the function \ToscaIn{Plus} and the list
\ToscaIn{(1,2,3)}.  As expected the result of
\ToscaIn{Main}, when evaluated, is \ToscaIn{(2,3,4)}

\subsection{\Tosca Language Features For Compilers} \label{sec:surface}

In addition to these general features, \Tosca adds features specific to compilers.
Some are rarely found in other programming languages, namely embedded programs,
syntactic variables, and specificity-ordered pattern matching, while the remaining
one, enumeration, is ubiquitous.  

\paragraph{Embedded programs.} One of the cornerstone \Tosca
features is the ability to embed program fragments written in the source programming language \PL
directly into \Tosca programs: inside rules.

As described in Section \ref{sec:use}, \Tosca takes
two inputs: the syntax and the semantics of a language \PL. The syntax is
needed for two reasons: first, as mentioned before, to parse input program written in \PL, 
but secondly also to parse program fragments embedded in the semantic description of \PL. 
%
\begin{example} \label{ex:embedinline} % 
Consider this rule:   
\begin{lstTosca}
rule InlineLetSimpleExpr( 
  toplevel⟦ let #VAR = #simple_expr ;; #toplevel ;; ⟧)
  → ...
\end{lstTosca}
\end{example} 
%
The special construct \ToscaIn{toplevel⟦ ... ⟧} tells the
\Tosca compiler to parse the embedded program using the grammar
rule named \antlrIn{toplevel}. Section~\ref{sec:normalization} 
gives an in-depth description on how this works.
%
\begin{example} \label{ex:expandinline} %
  The rule after expanding the embedded syntax
\begin{lstTosca}
rule InlineLetSimpleExpr( Let( #VAR, TimeExpr(CompExpr(PrimaryExpr(#simple_expr))), 
                               #toplevel) )
  → ...
\end{lstTosca}
\end{example}
%
There are several advantages to this approach. First, it makes
\Tosca programs more readable---as demonstrated in the examples
above---by reducing the code size significantly. Also, the
correspondence between language syntax and its semantics becomes more
apparent. Furthermore, embedded program fragments are sometimes more
resilient to some language syntax refactorings. For instance,
augmenting the \MiniML language with a new kind of expressions, such
as logical operators, has no effect on the rule shown in
Example~\ref{ex:embedinline}, whereas it changes the rule shown in
Example~\ref{ex:expandinline}.  On the other hand, changing
\antlrIn{';;'} to \antlrIn{';'} has the opposite effect, since it
impacts the concrete syntax of the language.  Nonetheless, we argue
that improved readability prevails over minor language updates.
%
\paragraph{Enumeration.} 
\emph{Enumeration} with associated values is the principal \Tosca data type.
In the literature enumerations are also called sum type, tagged union, 
variant type or recently just enumerations, like in Swift or Rust.%
\footnote{cf.~\href{https://developer.apple.com/swift/}{developer.apple.com/swift} 
and~\href{https://www.rust-lang.org/}{www.rust-lang.org}, respectively.}
It is well-established that enumeration is very well-suited for
 representing programming language syntax.
 Following the current trend, \Tosca enumerations are declared
 using the keyword \ToscaIn{enum}.  Value variants are separated
 by the character \ToscaIn{|}. To show the correspondence between
 language syntax and enumeration recall the \antlr grammar for 
 the non-terminal \antlrIn{simple_expr} defined in Example~\ref{ex:MiniML}:
%
\begin{lstANTLR}
simple_expr : VAR | 'true' | 'false' | INT
            | '(' expr ')' 
\end{lstANTLR}
%
I.e., a simple expression consists of either a variable represented by the
lexical token \antlrIn{VAR}, the literals \antlrIn{'true'} or
\antlrIn{'false'}, an integer or an \antlrIn{expr} surrounded by
parenthesis. 

\begin{example}

This grammar naturally maps to the following enumeration:
%  
\begin{lstTosca}
enum Simple_expr | VAR | TRUE | FALSE 
                   | INT | Expr
\end{lstTosca}

\end{example}
%
Notice the enumeration value names are slightly changed in order to
accommodate the \Tosca syntax. Grammar literals are mapped to 
all upper case name, dropping the surrounding quotes. Section~\ref{sec:parsing}
formalizes this mapping from \antlr grammar onto \Tosca.

\paragraph{Syntactic Variables.} Programming languages commonly define ways to
declare variables
with strict scoping rules associated to
them. Compiler writers rely on operations to determine when variables
are not used, e.g., for dead code elimination, or used only once, e.g., for in-lining, 
or for other purposes.
\Tosca provides
functionalities to deal with \emph{syntactic variables} and \emph{scoped
  syntactic variables}. \Tosca represents either kind of variables 
  the same way as variable (cf. Section~\ref{sec:language}): by an identifier starting 
  with a lower case character. For instance
\ToscaIn{var}, \ToscaIn{index}, or
\ToscaIn{inScope} are all valid syntactic variables.

Recall the \MiniML grammar in Example~\ref{ex:MiniML}:
%
\begin{lstANTLR}
toplevel : 'let' VAR '=' expr ';;' toplevel ';;'
          | expr ';;'
\end{lstANTLR}
%
The \ToscaIn{let} top level production is the typical use case for
using scoped syntactic variables, since there is a neat correspondence
between the semantics of \MiniML \antlrIn{'let'} and \Tosca scoped
variables.  By doing this, the incomplete inline \antlrIn{'let'} (cf.~
Example~\ref{ex:expandinline} and \ref{ex:embedinline}) becomes easy
to implement.
%
But first, in order to define scoped variables directly in the \antlr grammar
of \PL, we extended the \antlr language
with three new options: 
\begin{enumerate*} [label=\itshape(\roman*)]
\item \label{opt:one}   \antlrIn{<boundvar=x>},
\item \label{opt:two}   \antlrIn{<bound=x>}, and
\item \label{opt:three} \antlrIn{<symbol>}.
\end{enumerate*}
%
Option~\ref{opt:one} can be attached to any terminal
symbol, e.g., \antlrIn{VAR<boundvar=x>} to map 
this terminal symbol onto a bound variable. It also gives an internal id (like
\antlrIn{x}) so that it can be referenced in Option~\ref{opt:two}, \antlrIn{<bound=x>}, 
to start a new lexical scope in which the bound variable can occur.  
Option~\ref{opt:three} indicates which terminal needs to be mapped onto a syntactic variable
by looking for a bound variable of the same name, in the list of in-scope variables maintained
by the parser (cf. Section~\ref{sec:parsing}). If none is found, then the variable is considered free.

\begin{example} Let us rewrite the inline \antlrIn{'let'} example
using scoped variables. We start by rewriting the \MiniML \antlr grammar.
\begin{lstANTLR}
toplevel : 'let' VAR<boundvar=x> '=' expr ';;' 
                   toplevel<bound=x> ';;' | ...

simple_expr : VAR<symbol> | ...
\end{lstANTLR}

Next we change the corresponding \ToscaIn{enum Toplevel}:

\begin{lstTosca}
enum Toplevel | Let( Expr, [Simple_expr] -> Toplevel )
                | Expr( Expr )
\end{lstTosca}

Finally we replace \ToscaIn{VAR} in \ToscaIn{enum Simple_expr} 
by the keyword \ToscaIn{allows-variables}:

\begin{lstTosca}
enum Simple_expr | allows-variables | TRUE | FALSE 
                   | INT | Expr
\end{lstTosca}
\end{example}

Now the first \ToscaIn{TopLevel} enumeration value \ToscaIn{Let} 
takes two associated values: the let expression bound to \antlrIn{VAR} and 
the \antlrIn{toplevel} defining the scope of the \ToscaIn{let} variable.
Notice that the type \ToscaIn{Simple_expr} has also been updated to accept 
syntactic variable by adding \ToscaIn{allows-variables} keyword, which means
variables are valid values for \ToscaIn{Simple_expr} .

\begin{example} Finishing the implementation of \ToscaIn{InlineLetSimpleExpr} 
is now trivial:

\begin{lstTosca}
rule InlineLetSimpleExpr( 
  toplevel⟦ let #VAR = #simple_expr ;; #toplevel ;; ⟧)
  → #toplevel(#simple_expr)
\end{lstTosca}

\end{example}

This rule matches the \ToscaIn{let} construction binding simple expression,
and when the rule is evaluated, it substitutes all occurrences of \ToscaIn{\#VAR} 
in \ToscaIn{#toplevel} by \ToscaIn{#simple_expr}. 

On a final note, syntactic variables are not solely limited to encode programming language
variables. This mechanism can also be used everywhere there is a need
to associate a name to an entity referenced within a context. It
includes, but is not limited to, block labels, constants, functions,
modules, or packages.

\paragraph{Specificity-Ordered Pattern Matching.} Pattern matching is
frequently used in compiler generator to accomplish essentially two
main tasks: to dispatch on language construct to compile, and to
recognize specific program fragments to optimize. Several simple
dispatch example were given in Example~\ref{ex:evalminiml}. The
next example shows the second case: identifying program fragments
that can be optimized.
%
\begin{example} Compiling the \MiniML's \antlrIn{if then else}
can be optimized by two more specific rules 
and one fallback rule.
   
\begin{lstTosca}
rule CompExpr(expr⟦if true then #expr2 else #expr3⟧) 
  → CompExpr(#expr2)

rule CompExpr(expr⟦if false then #expr2 else #expr3⟧) 
  → CompExpr(#expr3)

rule CompExpr(expr⟦if #expr1 then #expr2 else #expr3⟧) 
  → ...
\end{lstTosca}	
\end{example}
%
Notice that the optimization is applied on the function \ToscaIn{CompExpr}
\emph{compiling} \MiniML.   
%
It is fairly easy to see that for more complex languages than \MiniML, troubleshooting issues, 
such as identifying missing or redundant cases, becomes harder. This is especially true when 
pattern matching is used to single out very specific program fragments which can grow quite large. 
It is also more difficult to identify when a pattern is more specific than another pattern. 
The \emph{specificity} compilation phase described in Section~\ref{sec:specification} 
is about providing a solution to these concerns.


%%% Local Variables:
%%% mode: latex
%%% TeX-master: "techreport"
%%% End:
